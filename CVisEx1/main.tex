\documentclass[letterpaper, 11pt]{article}

% Standard packages
\usepackage{amsmath, amsthm, latexsym, amssymb, graphicx, color, mathtools, geometry}

% Simplifies margin settings
\usepackage{geometry}
\geometry{margin=1in,headsep=.25in}

% Puts list item indicators in bold; makes flush with previous margin
\renewcommand\labelenumi{\bf\theenumi.}
\renewcommand\labelenumii{\bf\theenumii.}
\setlength\leftmargini{1.4em}
\setlength\leftmarginii{1.4em}

% Flexibility for headers and footers
\usepackage{fancyhdr}
\usepackage{datetime2}


\pagestyle{fancyplain}
\fancyhf{} %clear all header and footer fields
\cfoot{\bf \small Page \thepage}
\headsep 0.2in
\thispagestyle{empty}

\usepackage[pdftex]{hyperref}
\hypersetup{
    unicode=false,          % non-Latin characters in Acrobat's bookmarks
    pdftoolbar=true,        % show Acrobat's toolbar?
    pdfmenubar=true,        % show Acrobat's menu?
    pdffitwindow=true,      % page fit to window when opened
    pdftitle={My title},    % title
    pdfauthor={Author},     % author
    pdfsubject={Subject},   % subject of the document
    pdfnewwindow=true,      % links in new window
    pdfkeywords={keywords}, % list of keywords
    colorlinks=true,        % false: boxed links; true: colored links
    linkcolor=black,        % color of internal links
    citecolor=green,        % color of links to bibliography
    filecolor=magenta,      % color of file links
    urlcolor=blue           % color of external links
}

\renewcommand{\headrulewidth}{0pt}

\parindent 0in
\parskip 12pt

\begin{document}

\title{Homework Template}

\begin{center}
    {
        \large
        \bf
        CS-E4850 Computer Vision\\
        Exercise Round \#1\\
        Submitted by Chen\ Xu, ID 000000\\
        \today
    }
\end{center}

\bigskip

\textbf{Exercise 1. Homogeneous coordinates.}\\

\textbf{Solution}\\

a)
\begin{proof}
    The homogeneous coordinates of \textbf{x} is $(x\ y\ 1)^\top$, and $\textbf{l} = (a\ b\ c)^\top$,\\
    therefore the left-hand side of $ax+by+c=0$ can be written as the inner product of \textbf{x} and \textbf{l}. Thus,
    $$\textbf{x}.\textbf{l}=\textbf{x}^\top\textbf{l}=0$$
\end{proof}

b)
\begin{proof}
    $\because$ $\textbf{l}.(\textbf{l}\times\textbf{l}')=0$\\
    $\therefore$ the point $\textbf{x}=\textbf{l}\times\textbf{l}'$ is on the line \textbf{l}.\\
    Similarly, $\because$ $\textbf{l}'.(\textbf{l}\times\textbf{l})'=0$, \\
    $\therefore$ the point $\textbf{x}$ is on the line $\textbf{l}'$.\\
    The point \textbf{x} is on both \textbf{l} and $\textbf{l}'$, which means \textbf{x} is the intersection of \textbf{l} and $\textbf{l}'$.
\end{proof}

c)
\begin{proof}
    $\because$ $\textbf{x}.(\textbf{x}\times\textbf{x}')=0$\\
    $\therefore$ the line $\textbf{l}=\textbf{x}\times\textbf{x}'$ is through the point \textbf{x}.\\
    Similarly, $\because$ $\textbf{x}'.(\textbf{x}\times\textbf{x})'=0$, \\
    $\therefore$ the line $\textbf{l}$ is through the point $\textbf{x}'$.\\
    The point line \textbf{l} is through both points \textbf{x} and $\textbf{x}'$.
\end{proof}

d)
\begin{proof}
    From c), the line through both \textbf{x} and $\textbf{x}'$ can be written as $\textbf{l}=\textbf{x}\times\textbf{x}'$.\\
    $\because$ $\textbf{y}.\textbf{l} = (\alpha\textbf{x}+(1-\alpha)\textbf{x}').(\textbf{x}\times\textbf{x}')=\alpha\textbf{x}.(\textbf{x}\times\textbf{x}')+(1-\alpha)\textbf{x}'.(\textbf{x}\times\textbf{x}')=0$\\
    $\therefore$ the point $\textbf{y}=\alpha\textbf{x}+(1-\alpha)\textbf{x}'$ lies on the line \textbf{l}, which is through both points \textbf{x} and $\textbf{x}'$.

\end{proof}

\textbf{Exercise 2. Transformations in 2D.}\\
a) Use homogeneous coordinates and give the matrix representations of the following
transformation groups: translation, Euclidean transformation (rotation+translation),
similarity transformation (scaling+rotation+translation), affine transformation, projective transformation.\\
\begin{itemize}
    \item Translation:
          $$\begin{bmatrix}
                  1 & 0 & t_x \\
                  0 & 1 & t_y \\
                  0 & 0 & 1
              \end{bmatrix}$$
    \item Euclidean transformation (rotation+translation):
          $$\begin{bmatrix}
                  cos\theta & -sin\theta & t_x \\
                  sin\theta & cos\theta  & t_y \\
                  0         & 0          & 1
              \end{bmatrix}$$
    \item Similarity transformation (scaling+rotation+translation):
          $$\begin{bmatrix}
                  s\ cos\theta & -s\ sin\theta & t_x \\
                  s\ sin\theta & s\ cos\theta  & t_y \\
                  0            & 0             & 1
              \end{bmatrix}$$
    \item Affine transformation:
          $$\begin{bmatrix}
                  a_{11} & a_{12} & t_x \\
                  a_{21} & a_{22} & t_y \\
                  0      & 0      & 1
              \end{bmatrix}$$
    \item Projective transformation:
          $$\begin{bmatrix}
                  h_{11} & h_{12} & h_{13} \\
                  h_{21} & h_{22} & h_{23} \\
                  h_{31} & h_{32} & h_{33} \\
              \end{bmatrix}$$
\end{itemize}
b) What is the number of degrees of freedom in these transformations?
\begin{itemize}
    \item Translation:
          $2$
    \item Euclidean transformation (rotation+translation):
          $3$
    \item Similarity transformation (scaling+rotation+translation):
          $4$
    \item Affine transformation:
          $6$
    \item Projective transformation:
          $8$
\end{itemize}
c) Why is the number of degrees of freedom in a projective transformation less than the
number of elements in a 3 × 3 matrix?

The matrix $\begin{pmatrix}
        h_{11} & h_{12} & h_{13} \\
        h_{21} & h_{22} & h_{23} \\
        h_{31} & h_{32} & h_{33} \\
    \end{pmatrix}$
has 9 elements, but only their ratio matters, so the perspective transformation can be determined by 8 elements.


\textbf{Exercise 3. Planar projective transformation.}\\

\textbf{Solution}\\

a)
\begin{align*}
     & \textbf{l}^\top\textbf{x}=0                                                                    \\
     & \Rightarrow \textbf{l}^\top\textbf{H}^{-1}\textbf{H}\textbf{x}=0                               \\
     & \Rightarrow \textbf{l}^\top\textbf{H}^{-1}\textbf{x}'=0                                        \\
     & \therefore \textbf{l}'^\top = \textbf{l}^\top\textbf{H}^{-1}                                   \\
     & \Rightarrow \textbf{l}' = (\textbf{l}^\top\textbf{H}^{-1})^\top = \textbf{H}^{-\top}\textbf{l}
\end{align*}

b)
\begin{proof}
    Since $\pmb{x}_{1}$ and $\pmb{x}_{2}$ are not lying on the lines $\pmb{l}_{1}$ or $\pmb{l}_{2}$,
    Therefore:
    $$
        \pmb{l}_{i}^\top \pmb{x}_{j} \neq 0\ \forall i, j \in \{1, 2\}
    $$

    \begin{align*}
        I & = \frac{(\pmb{l}_{1}^\top \pmb{x}_{1})(\pmb{l}_{2}^\top \pmb{x}_{2})}{(\pmb{l}_{1}^\top \pmb{x}_{2})(\pmb{l}_{2}^\top \pmb{x}_{1})}                                                                                                     \\
          & = \frac{(\pmb{l}_{1}^\top \textbf{H}^{-1}\textbf{H}\pmb{x}_{1})(\pmb{l}_{2}^\top \textbf{H}^{-1}\textbf{H}\pmb{x}_{2})}{(\pmb{l}_{1}^\top \textbf{H}^{-1}\textbf{H}\pmb{x}_{2})(\pmb{l}_{2}^\top \textbf{H}^{-1}\textbf{H}\pmb{x}_{1})} \\
          & = \frac{(\pmb{l}_{1}'^\top \pmb{x}_{1}')(\pmb{l}_{2}'^\top \pmb{x}_{2}')}{(\pmb{l}_{1}'^\top \pmb{x}_{2}')(\pmb{l}_{2}'^\top \pmb{x}_{1}')} = I'                                                                                        \\
    \end{align*}
    $\therefore$ $I$ is invariant under projective transformation.
\end{proof}

Why similar construction does not give projective invariants with fewer number of points
or lines?

\textbf{Solution}

If there are fewer number of lines (1 line) and $I$ is constructed in such a way:
$$
    I = \frac{(\pmb{l}_{1}^\top \pmb{x}_{1})}{(\pmb{l}_{1}^\top \pmb{x}_{2})}
$$
Consider scaling $\pmb{x}_{1}$ by $\alpha$, $\pmb{x}_{2}$ by $\beta$, and $\pmb{l}_{1}$ by $\gamma$.
\begin{align*}
    \widetilde{I} & = \frac{(\gamma\pmb{l}_{1}^\top\alpha \pmb{x}_{1})}{(\gamma\pmb{l}_{1}^\top \beta\pmb{x}_{2})} \\
                  & = \frac{\alpha(\pmb{l}_{1}^\top \pmb{x}_{1})}{\beta(\pmb{l}_{1}^\top \pmb{x}_{2})} \neq I      \\
\end{align*}
which means $I$ is not invariant. Similarly, $I$ won't be invariant either if there are fewer number of points (one point).



%\begin{align*}
%& I'=\frac{n!}{k!(n-k)!}\\
%& \Rightarrow \textbf{l}^\top\textbf{H}^{-1}\textbf{H}\textbf{x}=0\\
%& \Rightarrow \textbf{l}^\top\textbf{H}^{-1}\textbf{x}'=0\\
%& \therefore \textbf{l}'^\top = \textbf{l}^\top\textbf{H}^{-1}\\
%& \Rightarrow \textbf{l}' = (\textbf{l}^\top\textbf{H}^{-1})^\top = \textbf{H}^{-\top}\textbf{l}
%\end{align*}




%Use exponential notation to show that
%\begin{itemize}
%\item $(1+i)^3 = -2-2i$.
%\item $2i(\sqrt{3}+i)(1+i\sqrt{3})=-8$
%\end{itemize}

%\textbf{Solution}\\
%By direct computation we observe that
%\begin{align*}
%(1+i)^3 &= xxx\\
%		&= yyy\\
%        & \vdotswithin{=} \text{ blah, blah, blah}\\
%        &= -2-2i.
%\end{align*}

%Proceeding to the second part, we compute

%etc.

\end{document}
