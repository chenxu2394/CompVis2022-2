\documentclass[letterpaper, 11pt]{article}

% Standard packages
\usepackage{amsmath, amsthm, latexsym, amssymb, graphicx, color, mathtools, geometry}

% Simplifies margin settings
\usepackage{geometry}
\geometry{margin=1in,headsep=.25in}

% Puts list item indicators in bold; makes flush with previous margin
\renewcommand\labelenumi{\bf\theenumi.}
\renewcommand\labelenumii{\bf\theenumii.}
\setlength\leftmargini{1.4em}
\setlength\leftmarginii{1.4em}

% Flexibility for headers and footers
\usepackage{fancyhdr}
\usepackage{datetime2}


\pagestyle{fancyplain}
\fancyhf{} %clear all header and footer fields
\cfoot{\bf \small Page \thepage}
\headsep 0.2in
\thispagestyle{empty}

\usepackage[pdftex]{hyperref}
\hypersetup{
    unicode=false,          % non-Latin characters in Acrobat's bookmarks
    pdftoolbar=true,        % show Acrobat's toolbar?
    pdfmenubar=true,        % show Acrobat's menu?
    pdffitwindow=true,      % page fit to window when opened
    pdftitle={My title},    % title
    pdfauthor={Author},     % author
    pdfsubject={Subject},   % subject of the document
    pdfnewwindow=true,      % links in new window
    pdfkeywords={keywords}, % list of keywords
    colorlinks=true,        % false: boxed links; true: colored links
    linkcolor=black,        % color of internal links
    citecolor=green,        % color of links to bibliography
    filecolor=magenta,      % color of file links
    urlcolor=blue           % color of external links
}

\renewcommand{\headrulewidth}{0pt}

\parindent 0in
\parskip 12pt

\begin{document}

\title{Homework Template}

\begin{center}
    {
        \large
        \bf
        CS-E4850 Computer Vision\\
        Exercise Round \#8\\
        Submitted by Chen\ Xu, ID 000000\\
        \today
    }
\end{center}
\bigskip
\textbf{Exercise 1. Face tracking example using KLT tracker.}\\
Run the example as instructed below and answer the questions.

a) Run Exercise8.ipynb

b) Run Exercise8.ipynb with a different input by changing the input to obama.avi: \\frames=faceTracker('obama.avi')

c) What could be the main reasons why most of the features are not tracked very long in case b) above?

\textbf{Answer:} The frame rate is not high enough when the camera starts shaking.

d) How could one try to avoid the problem of gradually losing the features? Suggest one or more improvements.

\textbf{Answer:} Increase the frame rate.

e) Voluntary task: Capture a video of your own face or of a picture of a face, and check that whether the tracking works for you. That is, replace the input video path in faceTrackingDemo.py with the path to your own video.

\bigskip
\textbf{Exercise 2. Kanade-Lucas-Tomasi (KLT) feature tracking.}
\begin{proof}
    \text{When the geometric warping \textbf{W} is a translation. \textbf{W} can be written as:}
    \begin{align*}
        \textbf{W}                                    & =
        \begin{bmatrix}
            x + p_1 \\
            y + p_2
        \end{bmatrix}                                    \\
        \therefore
        \frac{\partial\textbf{W}}{\partial\textbf{p}} & =
        \begin{bmatrix}
            1 & 0 \\
            0 & 1
        \end{bmatrix}
    \end{align*}
    Equation (10) in the paper becomes:
    \begin{align*}
        \Delta\textbf{p} & = (\sum{[\nabla I \frac{\partial \textbf{W}}{\partial \textbf{p}}]^\top}[\nabla I \frac{\partial \textbf{W}}{\partial \textbf{p}}])^{-1} \sum{[\nabla I \frac{\partial \textbf{W}}{\partial \textbf{p}}]^\top[T(\textbf{x})-I(\textbf{W}(\textbf{x};\textbf{p}))]} \\
                         & = (\sum{[\nabla I]^\top}[\nabla I])^{-1} \sum{[\nabla I]^\top[T(\textbf{x})-I(\textbf{W}(\textbf{x};\textbf{p}))]}                                                                                                                                                 \\
                         & =
        \begin{bmatrix}
            \sum{I_x I_x} & \sum{I_x I_y} \\
            \sum{I_x I_y} & \sum{I_y I_y}
        \end{bmatrix}^{-1}
        \sum{
        \begin{bmatrix}
            I_x \\
            I_y
        \end{bmatrix}
        [-I_t]
        }                                                                                                                                                                                                                                                                                     \\
                         & =-
        \begin{bmatrix}
            \sum{I_x I_x} & \sum{I_x I_y} \\
            \sum{I_x I_y} & \sum{I_y I_y}
        \end{bmatrix}^{-1}
        \begin{bmatrix}
            \sum{I_x I_t} \\
            \sum{I_y I_t}
        \end{bmatrix}
    \end{align*}
\end{proof}

\end{document}
